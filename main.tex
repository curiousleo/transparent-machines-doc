\documentclass[a4paper, oneside, 10pt]{amsart}

\usepackage[utf8]{inputenc}
\usepackage[T1]{fontenc}
\usepackage[english]{babel}
\usepackage[babel=true,kerning=true]{microtype}

\usepackage{amsmath, amssymb}
\usepackage{hyperref}
\usepackage{fancyvrb}
\usepackage{color}
\usepackage[final]{listings}

% Radicle code listing:
\lstset{
    language=Radicle,
    % numbers=left,
    basewidth=0.5em,
    xleftmargin=0ex
}

% Customize Verbatim environment:
\RecustomVerbatimEnvironment
    {Verbatim}{Verbatim}
    {framesep=2mm, frame=single, label=REPL, rulecolor=\color{black}, fontsize=\small}

\newcommand*\eg{e.g.\ }
\newcommand*\ie{i.e.\ }
\def\id{\mathrm{id}}

\begin{document}

\title{Future of Radicle: Transparent Machines}
\date{}
\author{James Haydon}

\thanks{\textsuperscript{$\dagger$}Monadic, \texttt{james@monadic.xyz}}

\begin{abstract}
This document presents a vision for a possible future of Radicle. The main goal
is to transform Radicle into a technology that powers an ecosystem of
\emph{transparent state machines}. This is a network of replicated state
machines running the Radicle interpreter which are monitored by their peers, to
ensure they abide by the Radicle semantics.
\end{abstract}

\maketitle

\tableofcontents

\section{Current state of Radicle}

Currently Radicle is composed of two main parts:
\begin{itemize}
\item
  A deterministic programming language that is designed in order to be able to
  specify state machines whose semantics may evolve over time.
\item
  A daemon which allows hosting and reading such machines on IPFS, using:
  \begin{itemize}
  \item
    The IPFS DAG to store a linked list of blocks containing the Radicle
    expressions which determine the machine,
  \item
    IPNS as a mutable reference to the head of the list,
  \item
    IPFS Pubsub for communication of suggested inputs from peers to the owner of
    a machine.
  \end{itemize}
\end{itemize}

This setup has a number of drawbacks.
\begin{itemize}
  \item \emph{Unweildy reprogrammability.} The current design of the language
    has not proved itself to allow for easy reprogrammability of machines.
  \item \emph{No guarantee of Radicle semantics.} The machine owner has complete
    control over what the IPNS link points to. In particular there are no
    guarantees that the owner will update the pointer in such a way that, over
    time, the sequence of Radicle expressions respects the \emph{append-only}
    property. In particular the machine owner can rewrite history, change
    semantics, remove and include new data, etc., all without detection by other
    peers. In short there is no guarantee the view of the machine evolves
    according to the rules of a Radicle state machine, as defined in the Radicle
    whitepaper \cite{radicle}. This voids a lot of the purported benefits of
    Radicle (``rule-based governance'').
  \item Because all mutations to the state of the machine are made visible
    through updates to the IPNS link, updates are associated with a single
    cryptographic key. In other words, Radicle machines are
    \emph{single-writer}. This is partially mitigated by the mechanism of
    suggesting inputs via IPFS Pubsub, but this still leaves two problems:
    \begin{itemize}
    \item \emph{Owner availability burden:} The owner of the machine must be
      online at the same time as the request is sent.
    \item \emph{Isolation:} Each machine is its own little universe. Even
      fundamental things like establishing pairings between usernames and public
      keys must be restarted from scratch on each machine.
    \item \emph{Potential censorship:} The owner has the opportunity to reject
      certain inputs if this might benefit them.
    \end{itemize}
\end{itemize}

This document aims to tackle these problems by proposing:
\begin{itemize}
  \item A P2P system for \emph{monitoring} Radicle machines which, while giving
    no guarantee that owners will follow the Radicle semantics, make the
    detection of (some forms of) misbehaviour evident at the protocol
    level. Peers simply ignore misbehaving machines.
  \item A system for inter-machine communication, backed by proofs of inclusion
    of outputs on other machines.
  \item A simpler mechanism for restricting inputs at the language level, based
    on an updatable transactor function, rather than eval-redefinition, along
    with a non-hyper static environment to allow for better reprogrammability.
  \item Finally the potential censorship problem is partially tackled by making
    censorship as detectable as possible by clients, so that users may gossip
    (off-protocol) about misbehaving machines and stop using them. We also
    discuss a web-of-trust type solution which may be implemented at a later
    stage.
  \item
    New multi-writer Radicle reference types, backed by CRDTs, which mitigate
    the owner-availability burden.
  \item
    Finally we discuss how we can extend the confidence in the transparency of a
    system to the UI presented to end-users through the use of \emph{gateways}.
\end{itemize}

\section{Transparent machines}\label{transparent-machines}

\subsection{Motivation}

The main goal of this proposal is to fix the most important problem highlighted
above: currently machines can misbehave in serious ways, and not respect the
Radicle semantics at all. By creating an ecosystem of machines and monitors,
allowing for the detection and broadcasting of misbehaviour, we aim to make
machines \emph{transparent}: owners may still misbehave but it will be obvious
to everyone else. Machines are hosted by people/organisations which want to
expose transparency in the ways they operate, so they have an incentive to stick
to the rules.

The primary application for Radicle will still be the
hosting/management/governing of important open source projects, but other
applications we have in mind are:

\begin{itemize}
\item
  Certificate authorities, Name registries, etc.
\item
  Package managers,
\item
  Adding transparency to the governance systems used by e.g.~political
  organisations,
\item
  Adding transparency to the interactions between organisations
  (e.g.~NGOs, private corporations, etc.) end-users.
\end{itemize}

Essentially any organisation that wants to make transparent to the world one of
the systems it uses to operate should be able to use Radicle to specify and host
a machine to realise this goal. Even governments have been exploring ways to
increase transparency in their operations in an effort to combat corruption and
increase the faith their citizens have in them (\eg
\cite{ukraine-transparency}), so there could be wide uses and demand for such an
ecosystem.

There are already several distributed systems in production in which
certain entities publish secure logs, so that third parties may monitor
their activities and check they are behaving according to rules everyone
has agreed to. For example,
\href{https://en.wikipedia.org/wiki/Certificate_Transparency}{Certificate
Transparency} is:
\begin{quote}
an Internet security standard and open source framework for monitoring
and auditing digital certificates. The standard creates a system of
public logs that seek to eventually record all certificates issued by
publicly trusted certificate authorities, allowing efficient
identification of mistakenly or maliciously issued certificates.
\end{quote}

Each such distributed system must define afresh the protocol and set of rules
that are to be followed. The \emph{Radicle network} proposes a single unified
protocol and monitoring system for all such ``transparent machines''. Such a
unification would have several benefits:
\begin{itemize}
\item \emph{Protocol re-use:} Having many systems use the same underlying
  protocol makes that protocol safer, more resources can be spent in auditing
  its code, fixing vulnerabilities, etc.
\item \emph{Code re-use:} domain-logic specifying rules is likely to be similar
  across domains and organisations. For example many domains will require
  processes for holding votes, electing leaders, managing user permissions,
  etc. By creating a set of ready-to-use packages each new organisation may
  launch a tailored transparent machine with minimal effort.
\item \emph{Larger, homogeneous network of monitors:} Once a machine is launched
  it immediately benefits from a large network on monitors. Different domains
  can now share monitor nodes, because each monitor just needs to understand the
  base Radicle semantics, not specific domain requirements.
\end{itemize}

The ideas presented in this document take inspiration from:
\begin{itemize}
\item This insightful blogpost:
  \href{https://pfrazee.github.io/blog/secure-ledgers-dont-require-proof-of-work}{Secure
    ledgers don't require proof-of-work}.
\item \emph{Transparent Certificate} in general, including:
  \begin{itemize}
  \item \href{https://www.certificate-transparency.org/how-ct-works}{How it
    works}.
  \item The latest
    \href{https://tools.ietf.org/html/draft-ietf-trans-rfc6962-bis-31}{RFC}.
  \item Some of the
    \href{https://github.com/google/trillian/tree/master/docs}{docs} of the
    \href{https://github.com/google/trillian}{Trillian} repository.
  \item A paper: \href{https://arxiv.org/pdf/1806.08817.pdf}{Aggregation-Based
    Gossip for Certificate Transparency}.
  \end{itemize}
\end{itemize}
It might help to review these documents to better understand this one.

The implementation may use pre-existing P2P technologies, \eg~the secure ledger
might use an IPLD~\cite{ipld} linked list (as Radicle currently does), a DAT
hypercore feed~\cite{dat} or a Trillian log~\cite{trillian}. We base the
following on a hash-linked list for simplicity, but note that Trillian in
particular offers a similar API with respect to fetching of data and proving
inclusion/forks, but uses a \emph{dense Merkle tree} instead, making many
operations much more efficient. The theoretical properties are essentially the
same though.

\subsection{Protocol}

Preliminary definitions:
\begin{itemize}
\item
  \emph{Radicle-lang} refers to the Radicle programming language~\cite{radicle}:
  a syntax of valid expressions and an deterministic \emph{interpreter} which
  maintains state (or produces an error) when ran over a sequence of
  expressions.
\item
  A (valid) \emph{Radicle sequence} is a sequence of Radicle-lang expressions
  which does not throw an error when evaluated in order by the Radicle-lang
  interpreter.
\item
  A \emph{secure ledger} (also called a ``blockchain'', but without the implied
  proof-of-work consensus protocol) is a list of records, called \emph{blocks},
  which are linked using cryptography. Each block contains a cryptographic hash
  of the previous block, and transaction data. In our case the transaction data
  will be a list of Radicle expressions. Each block is signed by the same
  private key, whose corresponding public $p$ key identifies the ledger as a
  whole. We will simply say \emph{ledger} to mean secure ledger in this
  document, and \emph{ledger for $p$} when we wish to specify the corresponding
  public key.
\item
  A ledger $x$ is a \emph{prefix} of another $y$, written $x \leq y$, if $y$ is
  obtainable from $x$ by appending zero or more blocks to $x$. In particular
  this implies that $x$ and $y$ both use the same public key. This forms a
  partial order on the set of ledgers.
\item Two secure ledger are \emph{consistent} if one is a prefix of the other,
  that is, if they are comparable for $\leq$.
\end{itemize}

The \emph{Radicle network} is composed of two sorts of nodes: \emph{machines}
and \emph{monitors}.

\subsubsection{Radicle ledgers}

The point of the ecosystem is to maintain Radicle state machines which are
determined by \emph{Radicle ledgers}. A block of a Radicle ledger for $p$ is a
record with the following fields:
\begin{itemize}
\item
  \texttt{expressions}: a hash for the list of Radicle expressions,
\item
  \texttt{outputs}: a Merkle tree hash for the list of outputs for all
  expressions of the machine from the start. This allows for compact proofs that
  an output was seen on a machine, and is used in \S\ref{machine-communication}.
\item
  \texttt{previousHash}: hash of the previous block, or $\bot$,
\item
  \texttt{hash}: hash of this block,
\item
  \texttt{signature}: a signature for the whole record (minus the signature
  field), for public key $p$.
\end{itemize}

Note that in order to keep the contents of the blocks lightweight, the
expressions are not present directly, only a hash of them. Replication of the
actual expressions is left to another layer of the Radicle network. For example
the hash could be an IPFS CID, and the lists of expressions replicated through
IPFS. In the following we will assume that a hash uniquely determines a list of
Radicle expressions.

The set $R_p$ of Radicle ledgers for the public key $p$ is the smallest subset
of the set of non-empty lists of blocks such that:
\begin{itemize}
\item
  The ledger with a single block $a$, singed for $p$, with
  $a_\mathtt{previoushash} = \bot$ and $a_\mathtt{expressions} = []$ is a member
  of $R_p$.
\item
  If $x \in R_p$ with last block $a$, and $b$ is a block such that:
  \begin{itemize}
  \item
    $b$ is signed for $p$,
  \item
    $b_\mathtt{previousHash} = a_\mathtt{hash}$,
  \item
    Appending the list of expressions corresponding to $b_\mathtt{expressions}$,
    to the concatenation of all the lists of expressions corresponding to the
    hashes in $x$ produces a valid Radicle sequence. Furthermore,
    $b_\mathtt{outputs}$ is the root of the Merkle tree of all the outputs
    obtained by running through this list of expressions,
  \end{itemize}
  then appending $b$ to $x$ produces a member of $R_p$.
\end{itemize}

\subsubsection{Machines}

A \emph{Radicle machine} (or simply \emph{machine}), is a node (identified by a
public key) which maintains a Radicle ledger. The machine responds to requests
for inclusion of expressions in the ledger, and also makes the ledger accessible
for monitoring by monitor nodes. To be considered \emph{valid} a machine must:
\begin{itemize}
\item[$(M_1)$]
  Respect the \emph{append-only} property:
  \begin{itemize}
  \item
    All the ledgers disseminated by the machine are consistent with one another.
  \item
    If ledger $s_0$ is made available before ledger $s_1$, then $s_0$ must be a
    prefix of $s_1$.
  \end{itemize}
\item[$(M_2)$] Respect Radicle semantics: a ledger is invalid if the underlying
  sequence of expressions throws an error, or the output hashes do not match the
  \texttt{outputHash}'s contained in the blocks. In general all ledgers
  disseminated must conform to the definition of a Radicle ledger given above.
\item[$(M_3)$] Not censor: The machine should respond to all valid requests. In
  particular the machine should respond to requests made by monitors for the
  ledger, and try to include any valid expressions that are sent to it by other
  peers.
\end{itemize}

Property $(M_3)$ is hard to state precisely for various reasons that are
discussed below.

A ledger that outputs a ledger that doesn't respect $(M_2)$ is
\emph{corrupt}. Note that because ledgers are meant to start with the same dummy
block, any two non-corrupt ledgers for the same key must have a common prefix.

\subsubsection{Monitors}\label{monitors}

Monitor nodes observe the ledgers output by certain machines and gossip between
themselves in order to detect misbehaviour.

A machine might misbehave in one of the following ways:
\begin{itemize}
\item[(1)]
  Fork the ledger, so as to present different views to different clients, or in
  order to re-write history.
\item[(2)]
  Output an invalid ledger, for example to be able to make invalid claims on
  other machines.
\item[(3)]
  Refuse to respond to monitoring requests, in order to cover up some other
  misbehaviour.
\item[(4)]
  Refuse to respond to expression inclusion requests (censoring).
\end{itemize}

In the case of (1) or (2), it will disseminate a proof of the misbehaviour as
widely as possible. Monitors counter-act (3) by gossipping amongst themselves,
so that inconsistencies in the responses from the machine can be
detected. Monitors also have to be careful about unmonitoring and remonitoring
machines. For example if a machine can't parse a log output by a machine it
might decide to stop monitoring it, but then pick up monitoring again a few
weeks later after the machine starts outputting valid logs again. But this sort
of behaviour presents a way for machines to cheat: should they wish to re-write
history they may start outputting slightly corrupt logs in the hope that most
monitors will stop monitoring, only to start again once history has been
re-written. For this reason monitors should treat even mild corruptions of the
log quite seriously, and when unmonitoring a machine should persist the
machine's ID so as to make sure to never start monitoring it again.

Provable misbehaviours of a machine:
\begin{itemize}
\item
  A machine has \emph{forked} if it has disseminated two inconsistent ledgers. A
  \emph{fork-proof} consists of two blocks $b_0$ and $b_1$ (signed by the
  machine's key) which contain identical hashes for a previous block
  $b$. Because of the assumed properties of cryptographic hashes and signatures,
  it is a assumed that the existence of a fork-proof proves that a machine has
  forked.
\item
  A machine is \emph{invalid} if it has disseminated a ledger whose contents is
  not a valid Radicle sequence. In this case a (minimal) \emph{invalidity-proof}
  is a block $b$ such that the previous block is still valid.
\end{itemize}
Taken together, fork-proofs and invalidity-proofs form the set of
\emph{misbehaviour-proofs}.

Some other behaviours can not be proven to other nodes:
\begin{itemize}
\item
  A monitor may detect that a machine is not desseminating the ledgers
  monotonically. For example it might encounter a ledger $x_0$ via gossiping
  only after this make a request for the ledger directly from the machine, for
  which it responds $x_1$. If $x_1 < x_0$ then the machine has not respected
  $(M1)$.
\item
  A monitor may detect that a machine is selectively applying some expressions
  and not others. See \S\ref{censorship-detection} on censorship detection.
\end{itemize}
In cases that a monitor has detected unprovable misbheaviour, it creates a it
creates a \emph{misbehaviour-description}, including for example signed Pubsub
messages and HTTP responses, etc, and it should should make this known to other
monitors and stop monitoring the machine, and never restart monitoring it.

More formally, to be a valid monitor, a node must:
\begin{itemize}
\item Maintain and make available to peers its \emph{dead-set} $D$; a set of
  machine IDs paired with either a misbehaviour-proof or a
  misbehaviour-description. Whenever it connects to another peer it should make
  this dead-set available to the peer, and ask for that peer's dead-set
  $D'$. Once checking the proofs, all elements of the dead-set that are
  accompanied by misbehaviour proofs should be added to its own dead-set. It is
  free to add the other elements to its dead-set too, depending on how much it
  trusts that monitor. A monitor should never remove elements from the dead-set.
\item
  Maintain and make available to peers its \emph{monitor-set} $M$; a set of IDs
  of machines it is currently monitoring. It is free to add or remove machines
  from this set as it sees fit, as long as $M \cap D = \varnothing$.
\item
  For each machine $m \in M$ in its monitor set, to:
  \begin{itemize}
  \item[(1)] Subscribe to the ledger updates from $m$.
  \item[(2)] Subscribe to ledger updates for $m$ broadcast by other monitors.
  \item[(3)] Maintain the current most up to date ledger for $m$:
    \begin{itemize}
    \item In normal operation all ledgers it receives from (1) and (2) should be
      consistent, and hence there is always a unique maximum. This maximum
      should be broadcast to all other monitors of $m$.
    \item Should it receive a ledger, either through (1) or (2), which is not
      consistent with the one it is currently maintaining, then the monitor has
      detected a fork. In that case it should construct the minimal fork-proof
      and broadcast it to all other monitors of $m$.
    \item Should it receive a corrupt ledger, it should construct a minimal
      invalidity-proof and broadcast this to all other monitors of $m$.
    \end{itemize}
  \item[(4)] If at any point it constructs a misbehaviour-proof for $m$, it must
    remove $m$ from $M$, and add it to $D$.
  \item[(5)] Detecting non-provable misbehaviours is more involved and
    optional. Possible techniques are discussed in the next section. If at any
    point it detects non-provable misbehaviour it should add $m$ to $D$ and
    broadcast a misbehavior-description to other peers.
  \end{itemize}
\end{itemize}

\hypertarget{censorship-detection}{%
\subsection{Censorship detection}\label{censorship-detection}}

Checking that machines respect the last rule $(M_3)$, no-censoring, is harder
since there isn't a proof which can be disseminated among the peers. As a first
step, censorship will not be blocked at the protocol level. Instead we will
focus of censorship being as \emph{detectable} as possible to the
end-users. Users are then encouraged to gossip amongst themselves (out-of-band),
and simply stop using any machines that seem to be censoring inputs, or have
done so in the past.

Deciding if an input is being censored can by difficult. Given a ledger, an
expression $e$ may be valid as the next input. However the machine might have
already decided to include other expressions in the next block, some of which
invalidate $e$. Even if $e$ is insertable at any position in the next block that
is formed by the machine, and the machine received the expression $e$ before
starting that block, it is unclear that the machine tried to censor $e$. Indeed
it might have received another input which was incompatible with $e$, and also
incompatible with another input it decided to include in the next block, but was
otherwise valid. It is even plausible that such inputs may be crafted for this
very reason by a malicious actor in an attempt to block a certain input. To
counter-act this we make $(M_3)$ more precise in the following way:
\begin{itemize}
  \item[$(M_3')$] Machines consider potential expressions for inclusion into the
    ledger in such a way that an expression is rejected only if there exists
    some index in the next block at which inserting that expression would form
    an invalid sequence.
\end{itemize}

One algorithm that achieves this is the naive greedy one: expressions are
considered one at a time in a random order, creating an increasing valid
sequence. If appending the next item to the currently valid sequence does not
create a valid sequence then it is rejected and never reconsidered.

The disadvantage of this options is that checking censorship is quite expensive:
one must try to insert an input at each location of a certain number of blocks,
so possibly this check would only kick in at certain monitor nodes when other
nodes have started to get many rejected expressions.

In cases where censorship would be particularly bad (e.g.~voting on something),
specialised Radicle data-structures could be used to make the tracking of
potential expressions more transparent. For example a special set could be
initialised with a validation function to specify exactly what sort of values
are to be accepted. Expressions submitted for inclusion bypass the root
validator and go straight to the set, as long as it remains ``open''. A special
command, or another predicate, ``seals'' the set, at which point it can be
processed, e.g.~to determine the result of the vote. Such a system removes any
ambiguity on why an expression might not be accepted according to the Radicle
semantics.

\subsection{Inter-machine communication}\label{machine-communication}

Since Radicle ledgers record a Merkle tree hash of the output list at each
block, it is possible to create a compact proof witnessing that an output was
seen on some machine. Radicle-lang will be equipped with a new primitive
$(\text{\texttt{assert-on-machine}} \ k \ o \ p)$ which takes two arguments:
\begin{itemize}
\item
  A public key $k$ identifiying a machine,
\item
  An output value $o$,
\item
  A proof $p$. This is composed of the signed block in which the output appears,
  and the Merkle proof for the specific output of that block.
\end{itemize}
The function returns a Boolean value, indicating if the proof is valid.

A machine can then depend on another by validating such proofs. For example,
let's assume a machine $R$ has come to be trusted as a username registration
service. $R$ accepts all requests to assign a username $u$ (a string) to a
public key $k$, as long as that username has not already been assigned to
another. For each such valid request it will output $[u \ k]$, a vector of
length 2. Another machine $P$, representing an important open source project,
may then allow users to use certain usernames as long as they have been claimed
on $R$. To claim a username on $P$, a user can include a proof $p$ of the
presence of the output $[u \ k]$ on $R$, and the code defining the validation
process on $P$ would make a call $(\text{\texttt{assert-on-machine}} \ R_\id
\ [u \ k] \ p)$ in order to verify that the user may claim this username.

For cases when the state of a machine is more complex, Radicle can be augmented
with primitives for creating and checking authenticated datatypes\footnote{The
  work for this has already been done in an unmerged pull-request.}. A machine
can then create an authenticated piece of data representing a part of its domain
state, and return the root hash as the output. Other machines can then use this
to create proofs stating various properties of this state.

\subsubsection{Linear time}

Using a Merkle hash actually has another benefit over referencing outputs
directly: one can require that one uses a relatively ``fresh'' version of the
referenced machine's state. For example, a more sophisticated version of $R$
might have various means of revocating usernames for various reasons. The
function \texttt{assert-on-machine} could optionally also check that the block
being referenced is not strictly before any previous block that was referenced
from the same machine. This prevents users using very old inputs as part of
their proofs. While the machine $P$ may experience a ``lag'' in its view of $R$,
time still flows in the sime direction.

\subsubsection{RPC}

It is even possible that a machine offer a sort of RPC functionality, by
accepting some special inputs corresponding to specific queries, and outputting
the result. Machines can even evaluate quite arbitrary expressions with the
guarantee that state is not modified by using the pattern to evaluate
\texttt{e}:
\begin{verbatim}
(catch 'rpc-result
  (do (def r e)
      (throw 'rpc-result r))
  (fn [r] [:rpc-query e r]))
\end{verbatim}
Since \texttt{catch} always restores the entire state as it was before entering
the body, the machine's state will not be modified. The output is
\texttt{[:rpc-query e r]}, which the calling machine can then use as proof that
\texttt{e} evaluates to \texttt{r} on this machine.

\section{Radicle-lang simplification/improvements}\label{radicle-lang-simplificationimprovements}

\subsection{Simpler input restriction}\label{simpler-input-restriction}

Eval redefinition is being used to restrict the allowed inputs of a chain. But
in fact most of the logic for rejecting inputs is done by validators, at the
moment placed at the start of each ``command'' function.

Futhermore once an eval-redefinition has taken place, this heavily restricts the
programming available in the machine (that's the point), but this is annoying
for experimentation at the REPL, crafting custom queries, etc. For example we
have the idea that users who wanted JSON responses could just query
\texttt{(to-json\ (list-patches))} instead of \texttt{(list-patches)}, but this
is not valid after eval-redefinition.

Here we propose that instead there is a special purpose function which holds the
machines root validation function \texttt{tx}; a validation function that is
used for all inputs to the machine. When an expression $e$ is submitted to the
machine, the atom \texttt{tx} is evaluated, and it is expected to be a
function. If so, this function is invoked on $e$. Whatever this invocation
returns is what's finally evaluated. Furthermore, the Radicle executable can be
configured to bypass the root-validator using a simple flag, for when one wants
to experiment with the machine at the REPL or craft specialised queries. Daemons
which are owner-mode for a machine should obviously never bypass the root
validator when accepting new inputs, but the \texttt{query} endpoint on a local
daemon would just execute expressions normally, without calling \texttt{tx}.

Even just for a simple counter-machine, the eval-redefinition spec of the
state-machine is not beginner friendly (omitting the prelude):
\begin{lstlisting}
(def i (ref 0))

(def eval
  (fn [input rad-state]
    (match input
      :increment
        (eval (quote (modify-ref! i (fn [x] (+ 1 x)))) rad-state)
      :getCounter
        (list (read-ref i) rad-state))))
\end{lstlisting}

This is what it would look like with a root-validator:
\begin{lstlisting}
(def i (ref 0))

(def tx
  (fn [expr]
    (if (eq? expr :increment)
      (modify-ref! i (fn [x] (+ 1 x)))
      (throw :invalid-input "Can only :inc!"))))
\end{lstlisting}

Note that querying has now been pushed out of the machine code altogether; if
one wants to query the machine for the current value one can just use the
expression \texttt{(read-ref! i)}.

The function \texttt{tx} is expected to perform most state-updates itself, and
return a value which evaluates to itself, a simple log of what sort of
transaction was processed. Only when new \emph{code} is to be added to the
machine, for example redefinition of \texttt{tx} itself, would the function
\texttt{tx} return that quoted code, to then be executed on the machine. One of
the advantages of this is that it removes the need for the \texttt{eval}
function altogether, which makes the semantics and static-analysis of Radicle
much easier. However without eval-redefinition, it might be necessary to add
macro-definition functionality.

Such validation functions can be layered, higher-order, etc., and this makes for
more flexible code-reuse between machines. In particular, because they can be
stateful, this allows for the creation of interesting ``middleware'', for
example for conducting votes, authenticating users, moderating comments, etc.

For example if the participants of a machine decide all inputs should be signed
by some public key, then they could import a standard higher-order validation
function, designed as follows. The function $(\text{\texttt{signed-input}} \ f)$
expects all inputs to be of the form:
\begin{lstlisting}
[:signed-input
  {:nonce "123"
   :machine-id "abc"
   :public-key pk
   :signature s
   :input i}]
\end{lstlisting}
It would check that the signature \texttt{s} is indeed a signature for
the rest of the dict, and then pass
\begin{lstlisting}
{:input i
 :author pk}
\end{lstlisting}
to the sub-validator $f$.

Higher-order validator can also be stateful. For example, a validator can manage
a vote for some binary decision. It would look for specific sorts of inputs,
\emph{proposal inputs}, and when it encounters one, start a vote on that
proposal (other sorts of inputs are simply passed through to some
sub-handler). At that point it then also listens for \emph{vote inputs} for that
specific proposal, maintaining in state how much votes each outcome has
accumulated. When it successfully processes a vote, it simply returns the
metadata:
\begin{lstlisting}
[:vote-accepted
  {:vote-for "123"
   :user "user-id"}]
\end{lstlisting}
When enough votes have come through for there to be a winner, that expression is
passed through to another sub-handler.

Such a system can for example enable most inputs (new issues, comments, etc.) to
go through a standard transactor, but important code changes to go through a
voting process before being evaluated.

\subsection{Non-hyper static environment}

We have created a few remote machines and they all seem to have the
following:
\begin{itemize}
\item
  A big chunk of ``domain'' state (e.g.~a dict of issues).
\item
  Smaller ``metadata'' state (e.g.~used up nonces, next ID, admin set,
  etc.).
\item
  A set of commands, doing a lot of validation and a little bit of
  mutation.
\end{itemize}

Some ways people might ``reprogram'' a machine:
\begin{enumerate}
\def\labelenumi{\arabic{enumi}.}
\item
  Fix some of the data (add a default field).
\item
  Tweak validation of an existing command (add more conditions). Maybe
  also tweak the mutation this command does.
\item
  Add a whole new command.
\end{enumerate}

Scenario (1) is probably quite common as part of an update which does
(2) or (3), because new features of changes to old features might demand
a modified data format. In any case it is probably quite easy, just a
\texttt{(modify-ref\ ...)}.

The first step of scenario (3) is quite easy: add a new function for the
new command. However you must then mutate the input-validator to accept
this new invocation. This might be tricky: the pertinent validator might
be under a few layers of higher order validators. If you are lucky the
validtor you want to mutate was put in a ref (to be dereferenced by
other validators) and you can mutate that directly.

In any case it seems that scenario (2) (tweaking existing functionality)
is the tricky one. Relaxing the hyperstatic environment seems like one
way of making this easier, since now you can just resubmit definitions
for the functions you want to change.

Non-hyperstatic would also allow for old definitions to get garbage
collected, whereas before this wasn't possible because an old definition
could still be in use in some deeply nested closure. Machine
participants would then be much more encouraged to update the modules in
use on a machine.

\section{CRDT data types}

\subsubsection{OrbitDB and IPFS log}

Here we will quickly describe how IPFS-log works, which is what all the
datatypes of OrbitDB are based on. The ideas of Multi-Writer DAT are quite
similar.

IPFS-log is a conflict-free replicated data structure (CRDT) that uses IPFS for
storage and replication. The log is formed of \emph{entries} which are documents
on IPFS, identified by CID. Each entry contains a set of CIDs of previous
entries, usually the latest ones the node was aware of when creating the new
entry. This forms a DAG of entries. By being given an entry (a \emph{head}), a
node can recursively resolve the pointers to other entries in order to access
the whole log. The entries contain CRDT log-operations and so despite the
ordering on the entries being only partial, the view that nodes have on the
order of the items in the log is eventually consistent.

IPFS-log allows restricting the write-access to a set of public keys. For the
moment this set can only grow (this is also the case with Multi-Writer DAT), but
the maintainers state that access revocation is currently on the development
roadmap.

\subsection{Input queue}

If the owner is offline then participants of a machine may pool the inputs they
would like to submit in a queue, which the owner can read from once they are
back online. If inputs \emph{always} go through such a decentralised
data-structure, then this also makes it easier to monitor which inputs are being
potentially censored by the machine.

IPFS-log could be used directly for this, but it would probably be more
efficient if each peer drops any inputs which are not valid according to its
current view of the machine.

\subsection{Radicle data types}

More ambitiously, we can imagine new Radicle reference types for storing data
that isn't as sensitive to ordering as the list of expressions defining the
semantics of the machine. For example an an ``append-only log'' could be
initiated as follows:

\begin{lstlisting}
(def valid-comment
  (fn [c]
    ...))
  
(def comments-crdt
  (new-crdt-log! valid-comment))
\end{lstlisting}

A validator is used to discard invalid comments. By default a log would not
restrict write-access, but this can be controlled by other functions.

Behind the scenes this would refer to a CRDT log either backed by IPFS-log, or
something similar. The name of the log is generated deterministically using the
machine's ID and a counter. \emph{Writing} to the log happens off-machine, with
special effectful functions added to Radicle. First one must obtain the name of
the log, and then one may append inputs to it:

\begin{Verbatim}[fontsize=\small]
> (crdt-log-append! comments {:author "james" :comment "This is so rad!"})
=> :ok
\end{Verbatim}

Even after people have added entries to the log, the machine will not see them,
that is \texttt{(read-crdt-log! comments)} will return \texttt{[]}. This is
because the machine's view must be updated explicitely by checkpointing a head:
$(\text{\texttt{crdt-log-checkpoint!}} \ h)$ where $h$ is a head of the
log. After such a checkpoint \texttt{(read-crdt-log! comments)} will return the
list of comments as of that checkpoint. Of course this means that the machine's
data doesn't actually change until a head is submitted, and this can only be
finalised by the owner. However query functions can use the latest head known by
the user:

\begin{Verbatim}[fontsize=\small]
> (some-query-fn!)
=> [...]
> (crdt-log-checkpoint! comments latest-head)
=> :ok
> (some-query-fn!)
=> [...] ;; more comments
\end{Verbatim}

A more sophisticated query function would make sure to query for the view of the
machine and the current view of the CRDT seperately, and then merge them in such
a way that e.g.~comments which are not yet visible to the machine are styled
differently or labelled as ``pending'', to make it apparent that these haven't
yet been checkpointed on the machine. Of course it is expected that the machine
owner checkpoints appropriate heads every once in a while, and if this isn't the
case then users may complain out of band, etc.

\section{UI gateways}

It's all well and good to have a transparent state machines but if the only way
to access it is through a complex network protocol end-users (especially
non-tech-savy ones) have no way of interacting with them. For this we would use
trusted \emph{portals}, servers which present a UI as specified in the machine
itself.

As a simple example, a machine can record and maintain an HTML template paired
with a machine query for populating the template placeholders. Each head of the
machine would then determine a unique piece of HTML, which can be hashed. This
pair of hashes can be sent to a monitor node for verification. This way the user
can be sure that the portal has not tampered with the UI in any way. Such
functionality could be implemented as a browser extension. The extension would
send the pair of hashes to known monitors of the machine (as specified in the
machine itself, or by the portal), and display an icon if they all confirm the
hash-pair, or block the page entirely if any one of them does not.

\begin{thebibliography}{9}

\bibitem{radicle} Monadic. Radicle. October 2018

\bibitem{ukraine-transparency}
  http://www.ua.undp.org/content/ukraine/en/home/blog/2018/the-expectations-and-reality-of-e-declarations.html

\bibitem{frazee-post} Blog post:
  \href{https://pfrazee.github.io/blog/secure-ledgers-dont-require-proof-of-work}{Secure
    ledgers don't require proof-of-work}

\bibitem{transparent-certificate}
  \href{https://www.certificate-transparency.org/how-ct-works}{Ceritficte Transparency: How it works}

\bibitem{dat}
  Ogden et.~al., Code for Science. 2008. ``Dat - Distributed Dataset Synchronization And Versioning''.

\bibitem{trillian}
  \href{https://github.com/google/trillian}{A transparent, highly scalable and cryptographically verifiable data store.}

\bibitem{ipld}
  \href{https://github.com/ipld/ipld}{IPLD -- InterPlanetary Linked Data}

\bibitem{orbitdb}
  \href{https://github.com/orbitdb/orbit-db}{OrbitDB -- A Peer-to-Peer Databases for the Decentralized Web}

\bibitem{ipfs-log}
  \href{https://github.com/orbitdb/ipfs-log}{IPFS-log -- Append-only log CRDT on IPFS}

\bibitem{multiwriter-dat}
  \href{https://www.datprotocol.com/deps/0008-multiwriter/}{Dat Protocol -- DEP-0008: Multi-Writer}

\bibitem{ct-gossip}
  \href{https://arxiv.org/pdf/1806.08817.pdf}{Aggregation-Based Gossip for Certificate Transparency}
  
\end{thebibliography}

\end{document}
